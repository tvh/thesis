\documentclass[a4paper,bibliography=totocnumbered,parskip=half]{scrbook}
\usepackage[utf8]{inputenc}
\usepackage[backend=biber]{biblatex}
\usepackage[english]{babel}
\usepackage{csquotes}
\usepackage{hyperref}
\usepackage{url}
\usepackage{scrhack}
\usepackage{listings}
\usepackage{graphicx}
\usepackage[T1]{fontenc}
\usepackage{amsmath}
\usepackage[obeyFinal]{todonotes}
\usepackage{blindtext}

% Kapitelüberschrift in der Kopfzeile
\usepackage[automark]{scrpage2} % Schickerer Satzspiegel mit KOMA-Script
\pagestyle{scrheadings}
\setheadsepline{.4pt}

\clubpenalty10000
\widowpenalty10000

\bibliography{accelerate}

\newcommand{\executeIffilenewer}[3]{
 \ifnum\pdfstrcmp{\pdffilemoddate{#1}}
 {\pdffilemoddate{#2}}>0
 {\immediate\write18{#3}}\fi
}
\newcommand{\includesvg}[2][]{
 \executeIffilenewer{#2.svg}{#2.pdf}
 {inkscape -z -D --file=#2.svg --export-pdf=#2.pdf}
 \includegraphics[#1]{#2.pdf}
}

\lstset{language=Haskell,
        basicstyle=\ttfamily,
        captionpos=b,
        inputencoding=utf8,
        extendedchars=\true} 


\begin{document}

\frontmatter

% Titelseite - ganz einfach
\titlehead{
  {\large Programming Languages and Compiler Construction}\\
  Department of Computer Science\\
  Christian-Albrechts-University of Kiel}
\subject{Master Thesis}
\date{\today}
%opening
\title{An LLVM Backend for Accelerate}
\author{Timo von Holtz}
\publishers{Advised By:\\Priv.-Doz. Dr. Frank Huch\\Assoc. Prof. Dr. Manuel M T Chakravarty}
\maketitle

\chapter*{Erklärung der Urheberschaft}

Ich erkläre hiermit an Eides statt, dass ich die vorliegende Arbeit ohne Hilfe Dritter und ohne Benutzung anderer als der angegebenen Hilfsmittel angefertigt habe;
die aus fremden Quellen direkt oder indirekt übernommenen Gedanken sind als solche kenntlich gemacht. 
Die Arbeit wurde bisher in gleicher oderähnlicher For in keiner anderen Prüfungsbehörde vorgelegt und auch noch nicht veröffentlicht.


\today
\begin{flushright}
\rule{6cm}{0.4pt} \\
Timo von Holtz
\end{flushright}
\clearpage
% oder auch manuell

% Verzeichnisse
\listoftodos
\tableofcontents   % Inhaltsverzeichnis
\listoffigures     % Abbildungsverzeichnis
\listoftables      % Tabellenverzeichnis
\lstlistoflistings % Abbildungsverzeichnis
\mainmatter

\chapter{Introduction}
\todo{Write Introduction}
\Blindtext

\chapter{Contributions}
\section{language-llvm-quote}
When writing a compiler using LLVM in Haskell there is a good tutorial on how to do it at \citeurl{diehl2014jit}.
It uses \citetitle{scarlet2013llvm} to interface with LLVM.
The general idea is to use a monadic generator to produce the AST on the fly.
This has some obvious drawbacks, as the code can get unreadable pretty quickly. \missingfigure{code example of monadic generation}

A solution is to use quasiquotation\cite{mainland2007quote}.
That way, one can write the llvm-ir directly, without having to manipulate the AST by hand.
This could also be done with a simple parser though.
The main advantage of quasiquotation is, the use of antiquotation.
They allow you to reference arbitrary Haskell values inside the quotation.

I implemented \citetitle{holtz2014quote}, a quasiquotation library for LLVM.
The design is inspired by \citetitle{mainland2007c}, which is also used in the cuda implementation of Accelerate.
I use \citetitle{gill1995happy} and \citetitle{alex}.
\todo{writeup of languale-llvm-quote}


\appendix

\backmatter
\printbibliography
\end{document}